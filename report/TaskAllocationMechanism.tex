\chapter{Task Allocation Mechanism}
The purpose of this project is to assess the benefit of dynamic, or adaptive, thresholds-based algorithms over static ones (whether it be the homogeneous or heterogeneous case). To do so, we introduce two performance metrics, one being the number of tasks processed within a time period and the other one being the average distance traveled per processed task. While the former gives us an idea of the average rate at which actions are performed, the latter embeds two notions at the same time: the distribution of workload among robots and the efficiency of the allocation in terms of distance from robots to tasks.

For the simulations to be comparable, we used the same Finite State Machine (FSM) for all simulations and all types of algorithms. The behavior of the robots could be described as follows. All the robots are initialized in the first state that consists in performing random turns and processing the image from the camera to determine if task needs to be handled. Once a task is picked, the robot changes state and goes directly towards the task. When the robot is close enough to the task, it stops for a given amount of time to process the task. Since the tasks are spawn at random locations, we wanted to maintain the randomness of the system. With this in mind, and as proposed in \cite{kalra}, we implemented an additional state that consists in performing random movements for a short period of time to thwart this tendency.

The framework used during the course of the project allowed us to test 3 main approaches. They could be divided in two categories, one being private (static case) and the other being public (dynamic case). On the one hand, we considered both homogenous (fixed and identical thresholds among robots) and heterogeneous (adative thresholds based on each robot's perception) cases. On the other hand, we took advantage of the local potential field emissions perceived by the robots to adjust the weights of each color stimulus to improve task allocation (though the thresholds are fixed).

\section{Private, Fixed-Threshold Algorithm}
[DESCRIBE THE ALGORITHM AND THE TUNING OF THE PARAMETERS]
\section{Private, Adaptive-Threshold Algorithm}
[DESCRIBE THE ALGORITHM AND THE INITIALIZATION OF THE PARAMETERS AND THE ADAPTATION PROCESS]
\section{Public, Fixed-Threshold Algorithm}
[DESCRIBE THE ALGORITHM AND THE INITIALIZATION OF THE PARAMETERS AND THE ADAPTATION PROCESS (BASED ON POTENTIAL FIELD)]

%MAYBE WE COULD TRY A PUBLIC, ADAPTIVE-THRESHOLD ALGORITHM BUT IT WOULD BE MUCH MORE COMPLICATED 
