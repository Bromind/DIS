\section{Conclusion}
With this project, we have presented three scalable, distributed and threshold-based approaches for task allocation in both static and dynamic cases with limited knowledge about the tasks and demand. We compared the efficiency and robustness of each approach in our case study involving 5 robots and 3 types of task in a large arena.

Robots controlled using probabilistic threshold based algorithms showed no real improvements over the deterministic case scenario, even for high steepness. The specialization of robots turned out not to be adapted to our case study as the number of robots is greater than the number of task categories and the different types of tasks are not equally represented at each time step (in both time and space). On the contrary, adaptation mechanism lowering the thresholds as the time in search for a task increases proved its efficiency in improving the robustness (and thus consistency) of the task allocation. Finally, we showed the benefits of the addition of communication to improve the task allocation mechanism. This allowed the robots to reconsider their choice of task to handle while heading towards it, hence preventing two robots from choosing the same task.

However, it should be noted that our study would greatly benefit from the implementing of an optimization algorithm such as Particle Swarm Optimization (PSO). Unfortunately, we were not able to implement it in time. We also believe that adding more randomness into the process would yield better overall performance. This could be done through the addition of a random time for spinning in search mode or random walk when the robots are too close (as suggested in \cite{3}). Finally, it could be interesting to study the consequences of allocating more time to search state for the robots to achieve a complete spin, and adapting their threshold based on what the perceived (vision, local perception) in a similar manner as \cite{3}.
