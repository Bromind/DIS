\section{Introduction}
This project deals with task allocation where there are multiple types of tasks. The robots will handle tasks that appear throughout the environment according to different task allocation strategies.

The main task allocation strategies can be divided into two categories: market-based and threshold-based approaches. A comparative study of both task allocation strategies is presented in \cite{1}. Though market-based task allocation is more efficient when information is accurate, the threshold-based approaches show similar results at a lower computational cost when the knowledge of the environment and the tasks is more limited. In our peculiar case where robot vision and communication capabilities are low, threshold-based approaches are more suited and will be studied through the course of this project.

As exposed in \cite{2}, the thresholds can either be considered fixed or variable according to various adaptation models. Moreover, the decision can either be deterministic or probabilistic (e.g. using a sigmoid function). Another possibility is to endow the robots with communication capabilities so that they can calibrate their thresholds based on received information from their neighbors \cite{3}.

In this project, we first propose static strategies where the weights of the stimuli are fixed and the thresholds are individual. We compare the homogeneous experiment, where all robots have the same threshold, to the heterogeneous case, where the robots adapt there thresholds depending on the time spent in search or by specializing in a peculiar task.

In an attempt to improve performance by avoiding multiple robots selecting the same task, we analyze the benefits of a dynamic strategy where each robot emits a virtual potential field for the task type it selected. The robots use the intensities of the potential field separately for each type of task and use this additional information to dynamically adjust the weights of the stimuli.
